\documentclass[12pt]{article}

\usepackage{sbc-template}

\usepackage{graphicx,url}

\usepackage[brazil]{babel}   
\usepackage[utf8]{inputenc}  

     
\sloppy

\title{Algoritmo Genéticos Paralelo: uma abordagem hierárquica}

\author{Derik Evangelista Rodrigues da Silva\inst{1}, Raphael Henrique Ferreira de Andrade\inst{1}, \\Eduardo Spinosa\inst{1}}

\address{Departamento de Informática -- Universidade Federal do Paraná
  (UFPR)\\
  Caixa Postal 19081 -- 81531-980 -- Curitiba -- PR -- Brasil
  \email{\{dersilva, rhfandrade, spinosa\}@inf.ufpr.br}
}

\begin{document} 

\maketitle

\begin{abstract}
  @TODO Abstract
\end{abstract}
     
\begin{resumo} 
  @TODO Resumo
\end{resumo}


\section{Introdução}

% O Algoritmo Genético (GA), frequentemente referenciado como \emph{algoritmos genéticos}, foi desenvolvido por John Holland na Universidade de Michigan, nos anos 70 (\cite{luke2009}).

Algoritmo Genetico (\emph{Genetic Algorithm} -- GA) são algoritmos de busca inspirados no processo de evolução e seleção natural \cite{goldberg1989} e tem tido grande sucesso em problemas de busca e de otimização, principalmente quando o espaços de busca é grande, complexo ou pouco conhecido, onde métodos de buscas convencionais (enumerativos, heurísticos, ...) não são apropriados \cite{herrera1998}. 

Um GA sequencial inicia-se gerando um conjunto de indivíduos para formar uma população inicial. Cada indivíduo representa uma possível solução do problema. Usando uma função de avaliação (chamada de função \emph{fitness}), mede-se a qualidade de cada indivíduo desta população. O cálculo do \emph{fitness} é, geralmente, o processo mais custoso de um GA \cite{paraleltax}. Seleciona-se aleatoriamente, então, um subconjunto de indivíduos desta população e neste é aplicado operadores estocásticos de seleção, mutação e cruzamento. Por fim, os indivíduos menos adaptados (ou seja, com pior \emph{fitness}) são descartados, para dar lugar a indivíduos mais bem adaptados.

% @TODO Colocar exemplos de sucesso!
Apesar do sucesso em muitas aplicações em diferentes domínios, existem, de acordo com \cite{paraleltax}, algums problemas que podem ser resolvidos com o uso de um Algoritmo Genético Paralelo (\emph{Parallel GA} -- PGA):

\begin{itemize}
  \item Para alguns tipos de problemas, o tamanho da população precisa ser muito grande, requerendo, consequentemente, uma grande quantidade de memória, podendo impossibilitar a execução eficiente em uma única máquina.
  \item O cálculo do \emph{fitness} consome muito tempo. Há registros na literatura de uma única execução consumindo mais de 1 ano de CPU.
  \item GA's sequencias podem ficar presos em regiões sub-ótimas, ficando impossibilitados de encontrar uma melhor solução. PGA's podem buscar em multiplos subespaços de busca em paralelo, e tem menos chance de ficar preso em regiões sub-ótimas.
\end{itemize}

O motivo mais importante para se estudar PGAs, ainda segundo \cite{paraleltax}, é que em muitos casos eles tem uma melhor performance do que os sequenciais, mesmo quando o paralelismo é simulado em uma máquina convencional. Segundo \cite{albasurvey}, eles atingem o objetivo ideal de se construir um algoritmo paralelo, onde o todo é melhor do que a soma das partes.

Este trabalho tem como objetivo comparar três tipos de arquiteturas de PGAs: múltiplas populações, arquitetura mestre-escravo e um híbrido de ambas, ou seja, uma combinação de múltiplas populaçõoes com mestre-escravo, aplicadas a otimização de funções. Além disso, compararemos os resultados com um GA sequencial convencional.



\bibliographystyle{sbc}
\bibliography{rel}

\end{document}
